\documentclass[conference]{IEEEtran}
\IEEEoverridecommandlockouts
% The preceding line is only needed to identify funding in the first footnote. If that is unneeded, please comment it out.
\usepackage{cite}
\usepackage{amsmath,amssymb,amsfonts}
\usepackage{algorithmic}
\usepackage{graphicx}
\usepackage{textcomp}
\usepackage{xcolor}
\def\BibTeX{{\rm B\kern-.05em{\sc i\kern-.025em b}\kern-.08em
    T\kern-.1667em\lower.7ex\hbox{E}\kern-.125emX}}
\begin{document}

\title{Analysis of a Pressure Separator Process \\
    in Gas Production and Demonstration of \\
    Pressure Relief Valve Devices \\
}

\author{\IEEEauthorblockN{Mai Thanh Hai\textsuperscript{1}, Dinh Hoang Son\textsuperscript{2}, Nguyen Bao Son\textsuperscript{3}}
    \IEEEauthorblockA{\textit{School of Electrical \& Electronic Engineering, Hanoi University of Science and Technology} \\
        Email: \textsuperscript{1}hai.mt210308@sis.hust.edu.vn, \textsuperscript{2}son.dh212417@sis.hust.edu.vn, \textsuperscript{3}son.nb212418@sis.hust.edu.vn}
}


\maketitle

\begin{abstract}
    Gas production is a critical industrial process that requires careful management to ensure both safety and efficiency. This paper focuses on the pressure separator process within gas production systems, highlighting its importance in separating high and low pressure gas. Utilizing an existing Piping and Instrumentation Diagram (P\&ID) as a practical example, the paper delves into the operational intricacies of the pressure separator and the role of pressure relief valve devices in maintaining system safety. The study aims to provide a detailed understanding of the processes and equipment involved, emphasizing the significance of proper pressure management in gas production.
\end{abstract}

\begin{IEEEkeywords}
    gas production, pressure separator process, pressure valve relief, safety
\end{IEEEkeywords}

\section{Introduction}

The gas production industry is a cornerstone of the global energy supply chain, necessitating stringent operational protocols to ensure both safety and efficiency. Among the critical components of this industry is the pressure separator process, which plays a pivotal role in distinguishing between high and low pressure gas streams. This paper presents an in-depth analysis of the pressure separator process, utilizing a practical example from an existing Piping and Instrumentation Diagram (P\&ID). Additionally, the study explores the functionality and significance of pressure relief valve devices, which are essential for maintaining system integrity and preventing hazardous overpressure scenarios. By examining these elements, the paper aims to enhance the understanding of pressure management practices within gas production systems, thereby contributing to improved operational safety and efficiency.


\section{Overview of Gas Production}

The production facility examined in this paper is operated by The General Oil \& Gas Operating Company, situated in Chemical City, Texas. This facility handles hydrocarbon fluids extracted from natural gas wells located on production platforms. These wells channel the production fluids into a main production header, which then delivers the feedstock to the facility. During the initial stage of the separation process (high pressure stage), the production fluids are directed into a high pressure separator where the liquid and gas components are separated under specific temperature and pressure conditions. The gas exiting the high pressure separator mainly consists of lighter hydrocarbons and requires no further treatment. This gas is transported via the export gas pipeline to nearby gas processing companies. In the subsequent stage of the separation process (low pressure stage), the liquid from the first stage is transferred to the low pressure separator, where it undergoes flashing at a designated temperature and pressure. The gas stream from the low pressure separator is compressed and then merged with the gas exiting the high pressure separator. The liquid from the low pressure separator is considered stabilized for processing and is pumped into the high pressure export liquid pipeline. The key equipment utilized in this process, includes a High Pressure Separator, Low Pressure Separator, Export Pump, and Gas Compressor.

\section{Four main processes in gas production}
Each of the four main processes revolves around a specific equipment within the studied gas production facility. Those equipment consist of the High Pressure Separator, Low Pressure Separator, Export Pump, and Gas Compressor. The following sections provide a detailed explanation of each process.

\subsection{High Pressure Separator}\label{HPS}

\subsection{Low Pressure Separator}\label{LPS}

\subsection{Export Pump}\label{EP}

\subsection{Gas Compressor}\label{GC}

\section{Pressure Relief Valve Hardware Device}

\subsection{Structure}

\subsection{Operating Principle}

\subsection{Classification}

\subsection{Application}

\subsection{Advantages and Disadvantages}


\section*{Acknowledgment}

The authors would like to sincerely thank Mrs. Dinh Thi Lan Anh for her guidance and support throughout the EE35550E course ``Process Control''during the 2024.1 semester at Hanoi University of Science and Technology.

\begin{thebibliography}{00}
    \bibitem{Kenexis} SA Gray and B Buckeye, ``Training Course Packet for Gas Production Facility ,'' Kenexis Consulting, June 2011.

\end{thebibliography}

\end{document}
